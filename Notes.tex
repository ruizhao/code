\documentclass[11pt, oneside]{article}   	% use "amsart" instead of "article" for AMSLaTeX format
\usepackage{geometry,listings}                		% See geometry.pdf to learn the layout options. There are lots.
\geometry{letterpaper}                   		% ... or a4paper or a5paper or ... 
%\geometry{landscape}                		% Activate for for rotated page geometry
%\usepackage[parfill]{parskip}    		% Activate to begin paragraphs with an empty line rather than an indent
\usepackage{graphicx}				% Use pdf, png, jpg, or eps� with pdflatex; use eps in DVI mode
								% TeX will automatically convert eps --> pdf in pdflatex		
\usepackage{amssymb}
\usepackage{hyperref}


\title{iOS development Learning Note}
\author{Rui Zhao}
%\date{}							% Activate to display a given date or no date
\usepackage{ifxetex}
%------------------Font specification-------------------%
% If you want to use font other than the default one,
% you MUST use XeLaTeX to compile.
\ifxetex
  \usepackage{fontspec}
  \usepackage{xunicode}
  % Use Monaco as default font, change down below (if you wish).
  \newfontfamily{\monaco}{Monaco} % Replace "Monaco" with another font name.
\else
  \newcommand{\monaco}{\ }
\fi
%-------------------------------------------------------%
\usepackage{color,xcolor}
\usepackage{listings}
\lstloadlanguages{[Objective]C}
\definecolor{nsclass}{RGB}{124,32,176}
\definecolor{atnotation}{RGB}{204,0,164}
\definecolor{import}{RGB}{128,70,30}
\definecolor{comment}{RGB}{0,140,0}
\definecolor{string}{RGB}{229,0,0}
\definecolor{method}{RGB}{70,0,134}
\definecolor{class}{RGB}{59,131,138}
\definecolor{custommethod}{RGB}{32,90,95}
\definecolor{number}{RGB}{56,0,225}
\definecolor{customgray}{RGB}{211,211,211}
\lstset{language=[Objective]C,tabsize=4, keepspaces=false,
    xleftmargin=0em,xrightmargin=-1em, aboveskip=1em, % Margin adjustment
    backgroundcolor=\color{customgray},    % Background color (Default:gray)
    frame=none,                            % Frame not needed
    breakindent=22pt,
    numbers=left,stepnumber=1,numberstyle=\tiny\color{black}\monaco,
    basicstyle=\fontsize{9pt}{1em}\selectfont\monaco,
   % commentstyle=\fontsize{9pt}{0.75em}\selectfont\monaco\color{comment},
   commentstyle=\color{comment},
    showspaces=false,
    flexiblecolumns=true,
    breaklines=true, breakautoindent=true,breakindent=4em,
    escapeinside={/*@}{@*/},
    morecomment=[s][\color{string}]{@"}{"},
    morecomment=[l][\color{import}]{\#},
    morecomment=**[s][\color{nsclass}]{NS}{];},
    morecomment=**[s][\color{nsclass}]{UI}{];},
    morecomment=**[s][\color{nsclass}]{NS}{(},
    morecomment=**[s][\color{nsclass}]{UI}{)},
    morecomment=**[s][\color{nsclass}]{UI}{*},
    morecomment=**[s][\color{nsclass}]{NS}{*},
    morecomment=*[s][\color{nsclass}]{UI}{\ },
    morecomment=*[s][\color{nsclass}]{NS}{\ },
}
% Down below, you can add your custom class names / method names as presented
% in your source code.
% For example, you have two custom class names called User and Person.
% You should add in the list User, Person
% The list of names should be seperated by commas, and no quotation
% marks are required.
\lstset{emph=[1]{  % <--Add your own Class Names before the percentage mark
       },emphstyle=[1]{\color{class}},
       moreemph=[5]{ % <--Add your own Method Names before the percentage mark
       },emphstyle=[5]{\color{method}},
}
\lstset{
    emph=[3]{@implementation,@synthesize, @interface, @property, @dynamic,
    @end, break, case, catch, class, copy, const, __finally, __exception,
    __try, const_cast, continue, private, public, protected, __declspec,
    default, delete, deprecated, dllexport, dllimport, do, dynamic_cast, else,
    enum, explicit, extern, if, for, friend, getter, goto, inline, mutable,
    naked, namespace, new, nil, NO, noinline, nonatomic, noreturn, nothrow,
    NULL, readonly, readwrite, register, reinterpret_cast, retain, return,
    SEL, selectany, self, setter, sizeof, static, static_cast, struct, super,
    switch, template, thread, throw, true, false, try, typedef, typeid,
    typename, union, using, uuid, virtual, void, volatile, whcar_t, while, YES,
    ATOM, BOOL, BOOLEAN, BYTE, CHAR, COLORREF, DWORD, DWORDLONG, DWORD_PTR,
    DWORD32,DWORD64, FLOAT, HACCEL, HALF_PTR, HANDLE, HBITMAP, HBRUSH,
    HCOLORSPACE, HCONV, HCONVLIST, HCURSOR, HDC, HDDEDATA, HDESK, HDROP,
    HDWP, HENHMETAFILE, HFILE, HFONT, HGDIOBJ, HGLOBAL, HHOOK, HICON,
    HINSTANCE, HKEY, HKL, HLOCAL, HMENU, HMETAFILE, HMODULE, HMONITOR,
    HPALETTE, HPEN, HRESULT, HRGN, HRSRC, HSZ, HWINSTA, HWND, INT, INT_PTR,
    INT32, INT64, LANGID, LCID, LCTYPE, LGRPID, LONG, LONGLONG, LONG_PTR,
    LONG32, LONG64, LPARAM, LPBOOL, LPBYTE, LPCOLORREF, LPCSTR, LPCTSTR,
    LPCVOID, LPCWSTR, LPDWORD, LPHANDLE, LPINT, LPLONG, LPSTR, LPTSTR, LPVOID,
    LPWORD, LPWSTR, LRESULT, PBOOL, PBOOLEAN, PBYTE, PCHAR, PCSTR, PCTSTR,
    PCWSTR, PDWORDLONG, PDWORD_PTR, PDWORD32, PDWORD64, PFLOAT, PHALF_PTR,
    PHANDLE, PHKEY, PINT, PINT_PTR, PINT32, PINT64, PLCID, PLONG, PLONGLONG,
    PLONG_PTR, PLONG32, PLONG64, POINTER_32, POINTER_64, PSHORT, PSIZE_T,
    PSSIZE_T, PSTR, PTBYTE, PTCHAR, PTSTR, PUCHAR, PUHALF_PTR, PUINT, PUINT_PTR,
    PUINT32, PUINT64, PULONG, PULONGLONG, PULONG_PTR, PULONG32, PULONG64, PUSHORT,
    PVOID, PWCHAR, PWORD, PWSTR, SC_HANDLE, SC_LOCK, SERVICE_STATUS_HANDLE,
    SHORT, SIZE_T, SSIZE_T, TBYTE, TCHAR, UCHAR, UHALF_PTR, UINT, UINT_PTR,
    UINT32, UINT64, ULONG, ULONGLONG, ULONG_PTR, ULONG32, ULONG64, USHORT,
    USN, VOID, WCHAR, WORD, WPARAM, WPARAM, WPARAM, char, bool, short, int,
    __int32, __int64, __int8, __int16, long, float, double, __wchar_t, clock_t,
    _complex, _dev_t, _diskfree_t, div_t, ldiv_t, _exception, _EXCEPTION_POINTERS,
    FILE, _finddata_t, _finddatai64_t, _wfinddata_t, _wfinddatai64_t,
        __finddata64_t,
    __wfinddata64_t, _FPIEEE_RECORD, fpos_t, _HEAPINFO, _HFILE, lconv, intptr_t,
    id, jmp_buf, mbstate_t, _off_t, _onexit_t, _PNH, ptrdiff_t,
    _purecall_handler, sig_atomic_t, size_t, _stat, __stat64, _stati64,
    terminate_function, time_t, __time64_t, _timeb, __timeb64, tm, uintptr_t,
    _utimbuf, va_list, wchar_t, wctrans_t, wctype_t, wint_t, signed
    },emphstyle=[3]{\color{atnotation}},
    moreemph=[4]{alloc, init, NSLog, sqrt, pow, cbrt, abs, fabs, powf
    },emphstyle=[4]{\color{method}}
}

\begin{document}
\maketitle
\tableofcontents

\section{MVC}
Controller has delegate of view. view don�t own the data they display. controller is the data source. 
Model wanna �tell� controller something, controller can listen to model(observation pattern)
\section{Start from an example}

\subsection{Vew heander}
\lstset{language=[Objective]C, breakindent=11pt, breaklines=true}


\begin{lstlisting}
@interface CalculatorViewController : UIViewController
//delegate,more strightforword way, but hide the view code
@property (weak, nonatomic) IBOutlet UILabel *display;
@end

\end{lstlisting}
Objective C support week reference and reference counting. 
Drag the view to make a delegation is cool, but we can't see the code for view.
\subsection{View implementation}

\begin{lstlisting}
//()make it private
@interface CalculatorViewController ()
@property (nonatomic) BOOL userIsInTheMiddleOfEnteringANumber;
@property (nonatomic,strong) CalculatorBrain *brain;
@end

@implementation CalculatorViewController
//_display is action location
@synthesize display = _display;
@synthesize brain = _brain;

- (CalculatorBrain *) brain {
    if (!_brain) {
    //syntax to new
        _brain = [[CalculatorBrain alloc] init];
    }
    return _brain;
}
\end{lstlisting}
make a member property and synthesize it can omit the getter and setter code, but we need to implement the getter if we wanna lazy initialize it. 
\begin{lstlisting}
//IBAcion is actually void,id is a pointer to unknow class of object
- (IBAction)digitPressed:(UIButton *)sender {
    NSString *digit = sender.currentTitle;
    //print the debug log
    NSLog(@"digit pressed = %@", digit);
    if (self.userIsInTheMiddleOfEnteringANumber) {
        self.display.text = [self.display.text stringByAppendingString:digit];
    } else {
        self.userIsInTheMiddleOfEnteringANumber = YES;
        self.display.text = digit;
    }
    
}
- (IBAction)operationPressed:(UIButton *)sender {
    if (self.userIsInTheMiddleOfEnteringANumber) [self enterPressed];
    double result = [self.brain performOperation:sender.currentTitle];
    NSString *resultString = [NSString stringWithFormat:@"%g", result];
    self.display.text = resultString;
    
    
}
 // don't need parameter here
- (IBAction)enterPressed {
    [self.brain pushOperand:[self.display.text doubleValue]];
    self.userIsInTheMiddleOfEnteringANumber = NO;
}

@end
\end{lstlisting}

\subsection{Model implemtation}
\begin{lstlisting}
@interface CalculatorBrain()
@property (nonatomic, strong) NSMutableArray *operandStack;
@end


@implementation CalculatorBrain

//_opreandStack is the actually name of the location
@synthesize operandStack = _operandStack;

- (NSMutableArray *) operandStack {
    if (_operandStack == NULL) {
        _operandStack = [[NSMutableArray alloc] init];
    }
    return _operandStack;

}

- (void) pushOperand:(double) operand{
    //boxing like java
    [self.operandStack addObject:[NSNumber numberWithDouble:operand]];
}

- (double) popOperand {
    NSNumber *operandObject = [self.operandStack lastObject];
    if (operandObject) [self.operandStack removeLastObject];
    return [operandObject doubleValue];
    
}

- (double) performOperation:(NSString *) operation{
    double result = 0;
    if ([operation isEqualToString:@"+"]) {
        result = [self popOperand] + [self popOperand];
    } else if ([@"*" isEqualToString:operation]) {
        result = [self popOperand] * [self popOperand];
    }
    
    return result;
}


@end
\end{lstlisting}
Define a class's public @interface and private @implementation in a .h and .m file respectively and a private @interface to .m file
\section{Objective C}
\subsection{Syntax, Language}
\subsubsection{Strong vs Weak}

strong:keep it in the heap until i don't point to it anymore;
weak:keep this as long as someone else points to it strongly, strong means ownership, weak means i just wanna get info of you.

\subsubsection{Comma vs Colon}
comma used when there is indeterminate parameters
\begin{lstlisting}
NSLog(@"digit pressed = %@", digit);
\end{lstlisting} 
\subsubsection{Initializers} 
Classes must designate an initializer for subclassers,(the reason is trying to avoid loop initialize) for subclassing reasons, init methods should be typed to return id(not statically typed)
\begin{lstlisting}
@implementation MyObject

//designated one. documentation
- (id) init
{
	self = [super init]; //call our super's designated initializer, convention
	if (self) {
		//initialize our subclass here
	}	
	return self;
}

-(id) initWithString (NSString *) str
{
        //do we have to do the assignment here?
	self = [self init];
	return self;
}

\end{lstlisting}

\subsubsection{Dynamic Binding}
All object are allocated in the heap, so always use a pointer, id itself is a pointer.
we can cast a pointer to any type and i will crash if it doesn't have the method we expected.

\begin{lstlisting}
 // call NSString's class method. not sure why Objective C don't just use the class name
 if ([obj isKindOfClass:[NSString class]])
\end{lstlisting}
\subsubsection{Introsepction}
Method testing
\begin{lstlisting}
if ([]obj responseToSelector:@selector(shoot)]) {
	[obj shoot];
}
\end{lstlisting}
SEL is type for selector

\begin{lstlisting}
SEL shootSelector = @selector(shoot);
SEL moveToSelector = @selector(moveTo:withPenColor);
[obj performSelector:shootSelector];
[array makeObjectPerformSelector:shootAtSelector withObject:target];
//in UIButton. 
[button addTarget:self action:@selector(digitPressed:) ...];
\end{lstlisting}

\section{Foundation Framework}
\subsection{NSObject}
Base class for pretty much every object in the iOS SDK, implements introspection methods, etc.

\subsection{NSString}
Immutable! Unicode!

\subsection{NSMutableString}
mutable one still need to go to heap and allocating memories. and it's not optimized like immutable one. 

\subsection{NSNumber}
Object wrapper around primitive types like int, float, double etc. so we can throw them into collection(NSArray or NSDictionary)
\begin{lstlisting}
NSNumber *num = [NSNumber numberWithInt:36];
\end{lstlisting}

\subsection{NSValue}
Generic object wrapper for other non-object data types. (c Struct)

\subsection{NSData}
"Bag of bits"

\subsection{NSArray}
Ordered collection of objects.
Immutable. make sense because we don't wanna screw things up after give the pointer to others. 
most language's array is mutable but we can use lock to make sure we don't change it when somebody else is changing it. 


\subsection{NSMutableArray}

\begin{lstlisting}
+(id) arrayWithCapacity:(int) initialSpace;
-(id) copy;//returns an NSAarry.
\end{lstlisting}
\subsection{NSDictionary}
Immutable hash table.

\subsection{NSSet}
set, unordered. i guess it's more efficient than c++'s set because that one use red-black tree.
\begin{lstlisting}
-(int) count;
-(BOOL) containsObject:(id) anObject;
-(id) anyObjects;
-(void) makeObjectsPerformSelector:(SEL) aSelector;
\end{lstlisting}

\subsection{NSMutableSet}
Mutable version of NSSet.


\begin{lstlisting}
- (void) addObject:(id) anObject;// does nothing if object that isEqual:anObject is already in 
- (void) removeObject:(id) anObject.
\end{lstlisting}

\subsection{NSOrderedset}
Immutable ordered collection of distinct object. 

\subsection{NSMutableOrderedSet}
Mutable version of NSOrderedSet

\subsection{Enumeration}
Looping through members of a collection in an efficient manner.
\subsubsection{array}
\begin{lstlisting}
NSArray *myArray = [[NSArray alloc] init];
for (NSString *string in myArray) { //no way for compiler to know what myArray contains
	double value = [string doubleValue]; // crash here if string is not an NSString
}
\end{lstlisting}
\subsubsection{dictionary}
\begin{lstlisting}
for (id key in myDictionary) {
	id value = [myDictionary objectForKey:key];
}
\end{lstlisting}

\subsection{Property List}
A collection of collections. i think it simply has to have an order so we can read and write it to somewhere
it is any graph of objects containing only the following classes:
\begin{lstlisting}
NSAarray, NSDictionary, NSNumber, NSString, NSDate, NSData.

[plist writeToFile:(NSString *) path atomically:(BOOL)]; // plist is NSArray or NSDictionary
\end{lstlisting}



\subsection{NSUserDefaults}
Lightweight storage of Property List.
\begin{lstlisting}
[[NSUserDefaults standardUserDefaults] setArray:rvArray forKey:@"RecentlyViewed"]
\end{lstlisting}


\subsection{Continue our example}
We use a mutable array to save the operand and operators.
\begin{lstlisting}
@interface CalculatorBrain()
@property (nonatomic, strong) NSMutableArray *programStack;
@end
\end{lstlisting}
we push everything to the array, runProgram takes program/array as a parameter and creates an mutable copy from it. 
\begin{lstlisting}
- (void) pushOperand:(double) operand{
    //boxing like java
    [self.programStack addObject:[NSNumber numberWithDouble:operand]];
}
- (double) performOperation:(NSString *) operation{
    [self.programStack addObject:operation];
    return [CalculatorBrain runProgram:self.program];
}
+(double)runProgram:(id)program{
    NSMutableArray *stack;
    if ([program isKindOfClass:[NSArray class]]) {
        stack = [program mutableCopy];
    }
    return [self popOperandOffStack:stack];
    
}
\end{lstlisting}
calculate the result, each time we got the mutable array, we recursively call the popOperandOffStack to get the result.  drawback: we need to do the recursive call each time. so we recalculated lots of stuff. 
\begin{lstlisting}
+(double) popOperandOffStack:(NSMutableArray *)stack {
    double result = 0;
    id topOfStack = [stack lastObject];
    if (topOfStack) [stack removeLastObject];
    if ([topOfStack isKindOfClass:[NSNumber class]]){
        result = [topOfStack doubleValue];
    } else if ([topOfStack isKindOfClass:[NSString class]]) {
        NSString *operation = topOfStack;
        if ([operation isEqualToString:@"+"]) {
            result = [self popOperandOffStack:stack] + [self popOperandOffStack:stack];
        } else if ([@"*" isEqualToString:operation]) {
            result = [self popOperandOffStack:stack] * [self popOperandOffStack:stack];
        }
    }
    
    return result;
}
\end{lstlisting}
\section{Views}
A view i.e. UIView subclass, is an rectangle on the screen. a view has only one superview. but can have different number of subviews. 
UIWindow is the UIView at the top of view hierarchy, only have one UIWindow in an iOS application.
\begin{lstlisting}

\end{lstlisting}
\subsection{CGFloat, CGPoint,, CGSize, CGRect}
Just a floating point number, always use it for graphics. 
\begin{lstlisting}
CGRect aRect = CGRectMake (4.5, 3.5, 3,3);
aRect.size.height+=45;
aRect.origin.x+=30;
\end{lstlisting}

\subsection{Coordinates}
(0.0) is the upper left, Units are "points" not pixels. for portable reason.  UIView has a property to tell.
\begin{lstlisting}
@property CGFloat contentScaleFactor;//returns pixels per point on the screen this view is on
\end{lstlisting}
View has 3 properties about size and location
\begin{lstlisting}
@property CGRect bounds;// adjust itself on screen
@property CGPoint center;// the center of your view in your superview's coord.
@property CGRect frame;// a rectangle in your superview's coordinate space which entirely contains your view's bounds.size;
\end{lstlisting}
\subsection{Drawing is fun}
Graphic is expensive, don't do premature optimization on other stuff as long as the code is clean;
\begin{lstlisting}
-(void) drawRect:(CGRect) aRect; // newer call drawRect;
\end{lstlisting}
We can use Core Graphics framework to implement drawRect. The API is C. 
The context determines where your drawing goes.
\begin{lstlisting}
CGContextRef context = UIGraphicsGetCurrentContext();
[[UIColor greenColor] setFill];
[[UIColor redColor] setStroke];
CGContextDrawPath(context, kCGPathFillStroke);//kCGPathFillStroke is a constant
CGContextBeginPath(context);//Begin the path
CGContextMoveToPoint(context, 7,5);
CGContextAddLineToPoint(context, 1,2);
CGContextClosePath(context);
\end{lstlisting}
we can build a path and reuse it later. CGPath.
Transparency is not Cheap. we can use UIGraphicsPushContext(ctxt) and popContext to keep graphics state liner. 
\subsection{Drawing Text}
UILabel.
UIFont can get a font. NSString are defined in UIKit via a mechanism called "categories".
\begin{lstlisting}

\end{lstlisting}
\subsection{Drawing Image}
Create a UIImage object from a file in your Resources folder
\begin{lstlisting}
UIImage *image = [UIImage imageNamed:@�foo.jpg�];
\end{lstlisting}
Or create one from a named file or from raw data
\begin{lstlisting}
UIImage *image = [[UIImage alloc] initWithContentsOfFile:(NSString *)fullPath];
UIImage *image = [[UIImage alloc] initWithData:(NSData *)imageData];
\end{lstlisting}

\section{Debugging}
\subsection{gdb}
\begin{lstlisting}
gdb print [self operand]
gdb print self
\end{lstlisting}
we can overwrite the description, so the print self will print that string.

\section{Protocols}
Protocols is actually interfaces in other languages, and "interface" in Objective C is "public.
\subsection{Basic}
\begin{lstlisting}
@protocol Foo <Other, NSObject> // implementors must implement Other and NSObject too
- (void)doSomething; // implementors must implement this (methods are @required by default) @optional
- (int)getSomething; // implementors do not have to implement this
- (void)doSomethingOptionalWithArgument:(NSString *)argument; // also optional
@required
- (NSArray *)getManySomethings:(int)howMany; // back to being �must implement� 
@property (nonatomic, strong) NSString *fooProp; // note that you must specify strength
@end				// (unless it�s readonly, of course)

\end{lstlisting}
\subsection{Why}
Number 1 use of protocols in iOS: delegates and  sources, This assumes that the object serving as delegate will outlive the object doing the delegating. Especially true in the case where the delegator is a View object

\section{UIGestureRecognizer}
There are two sides to using a gesture recognizer
1. Adding a gesture recognizer to a UIView to ask it to recognize that gesture.
2. Providing the implementation of a method to �handle� that gesture when it happens.
\begin{lstlisting}

\end{lstlisting}
\subsection{adding}
Adding a gesture recognizer to a UIView from a Controller
\begin{lstlisting}
 - (void)setPannableView:(UIView *)pannableView
      {
          _pannableView = pannableView;
          UIPanGestureRecognizer *pangr =
              [[UIPanGestureRecognizer alloc] initWithTarget:pannableView action:@selector(pan:)];
          [pannableView addGestureRecognizer:pangr];
}
\end{lstlisting}


\begin{lstlisting}
- (void)pan:(UIPanGestureRecognizer *)recognizer
{
if ((recognizer.state == UIGestureRecognizerStateChanged) ||
	(recognizer.state == UIGestureRecognizerStateEnded)) {
	CGPoint translation = [recognizer translationInView:self];
	@property called origin self.origin = CGPointMake(self.origin.x+translation.x, self.origin.y		+translation.y); 
	[recognizer setTranslation:CGPointZero inView:self];
	} 
}
\end{lstlisting}
\subsection{Enumeration}
\begin{lstlisting}

\end{lstlisting}


\subsection{Enumeration}
\begin{lstlisting}

\end{lstlisting}
\subsection{Enumeration}
\begin{lstlisting}

\end{lstlisting}
\subsection{Enumeration}
\begin{lstlisting}

\end{lstlisting}
\begin{lstlisting}
\end{lstlisting}

\subsection{}






















\end{document}  